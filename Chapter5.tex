\chapter{Results and Discussions}

This chapter evaluates the performance of the proposed Vehicle Congestion Control and Management System through a combination of simulation and real-time testing. Key outcomes include comparative analysis between traditional fixed-time traffic signals and our intelligent system, improvements in wait time, and responsiveness to emergency vehicles. Real-time video detection, sensor data processing, and web-based dashboard monitoring were used to validate system effectiveness.

\section{Contents of this Chapter}
\begin{enumerate}
    \item Simulation Results on Traffic Flow Optimization
    \item Real-Time Sensor and Camera-Based Experiments
    \item Emergency Vehicle Prioritization Testing
    \item Comparative Analysis with Static Signal Control
    \item Key Inferences and Observations
\end{enumerate}

\section{Simulation Results Using Traffic Scenarios}

Simulated traffic flow scenarios were generated using Python scripts to mimic high-density junction activity. Parameters such as vehicle density, traffic queue length, and signal wait time were monitored.

The intelligent system analyzed density from ultrasonic and IR sensors, along with camera-based vehicle counts. Signal durations were dynamically assigned based on computed vehicle flow using the equation:

\[
\gls{flow} = \gls{density} \times \gls{speed}
\]

These simulations were benchmarked against a traditional system operating on fixed-timer logic.

\section{Real-Time Experiment Results Using Sensor and Camera Setup}

A scaled prototype was constructed with ATmega2560, ESP8266, and high-resolution cameras at a 4-lane intersection model. Sensors captured real-time traffic data and camera streams were processed using YOLOv5.

The following were observed:
\begin{itemize}
    \item Accurate vehicle count using live video with >90\% detection rate.
    \item Real-time ultrasonic measurements effectively predicted traffic congestion levels.
    \item IR sensors reliably detected presence for default rotation in low-density cases.
\end{itemize}

\section{Emergency Vehicle Prioritization Test}

To simulate emergency situations, ambulance images were introduced in camera feed, or RF signal was manually triggered. Upon detection:
\begin{itemize}
    \item GPIO-controlled relays overrode ongoing signal cycles.
    \item The respective lane received immediate green light.
    \item Normal cycle resumed once the emergency vehicle passed.
\end{itemize}

This proved the system’s real-time responsiveness to emergency conditions and improved travel time for ambulances.

\section{Comparative Analysis}

The following table compares average wait time at an intersection for three different time periods in a day:

\begin{table}[H]
\centering
\caption{Wait Time Comparison: Fixed vs Intelligent System}
\label{tab:waittime}
\begin{tabular}{|p{4cm}|c|c|}
\hline
\textbf{Time of Day} & \textbf{Fixed Timer System (sec)} & \textbf{Proposed Intelligent System (sec)} \\
\hline
Morning Peak (8 AM) & 95 & 42 \\
Midday (12 PM) & 60 & 38 \\
Evening Rush (6 PM) & 105 & 49 \\
\hline
\end{tabular}
\end{table}

Key observations:
\begin{itemize}
    \item Overall wait time reduced by 40--55\% during peak hours.
    \item Emergency vehicle delay reduced to near-zero in test conditions.
    \item Web dashboard provided clear visualization for congestion heatmaps.
\end{itemize}

\section{Key Inferences and Observations}

\begin{itemize}
    \item The adaptive traffic control algorithm significantly outperformed traditional static systems.
    \item The fusion of camera-based computer vision and physical sensors enabled accurate congestion analysis.
    \item Machine learning-based prediction allowed proactive signal assignment.
    \item Emergency vehicle prioritization logic helped simulate real-life scenarios effectively.
\end{itemize}

\vspace{0.5cm}

The results validate that the proposed system is not only feasible but scalable to real-world intersections. In the next chapter, we conclude the project and outline enhancements for large-scale deployment.